\documentclass[11pt]{article}
\parindent 0pt
\usepackage[a4paper,left=0.75in,right=0.75in,top=1in,bottom=0.75in]{geometry}

% Upload packages
\usepackage{amsmath,amscd,amsbsy,amsfonts}
\usepackage[shortlabels]{enumitem}
\usepackage{fancyhdr}
\usepackage{lipsum}
\usepackage{setspace}
\usepackage{mathpazo}
\usepackage{color}
\usepackage[dvipsnames]{xcolor}

% Define some colors for convenience
\def\blue{\color{Turquoise}}
\def\purple{\color{Rhodamine}}
\def\white{\color{White}}
\def\orange{\color{Dandelion}}
\def\black{\color{Black}}

% Change spacing
\setstretch{1.25}

% Set page style to fancy
\pagestyle{fancy}

% Clear all header and footer fields
\fancyhf{}

% Set the right side of the footer to be the page number
\rfoot{\white \thepage}

% Set the header
\lhead{\white\textbf{MATH101. First-Year Math \hfill Problem Set 0}\\
GPT Chat, gc1234\\
\vspace{-2.5mm}
\noindent\white\hrulefill}

% Create new commands for frequently used commands
\newcommand{\Var}{\operatorname{Var}}
\newcommand{\Cov}{\operatorname{Cov}}
\newcommand{\Exp}{\operatorname{Exp}}
\newcommand{\R}{\mathbb{R}}
\newcommand{\Q}{\mathbb{Q}}
\newcommand{\Z}{\mathbb{Z}}
\newcommand{\calF}{\mathcal{F}}
\newcommand{\qed}{\hfill\square}

% Set font (typewriter style)
\usepackage{inconsolata}
\renewcommand*\familydefault{\ttdefault}
\usepackage[T1]{fontenc}
% This is one fo the few typewriter fonts that does not violate margins

% Define problem and solution commands
\def\pb#1{{\blue \bf Problem #1.}\vskip 1pt \white}
\def\sol{\purple \textbf{Solution:}\hskip 8pt \white}

%%%%%%%%%%%%%%%%%%%%%%%%%%%%%%%%%%%%%%%%%%%%%%%%%%%%%%%%%%%%%%%%%%%%%%%%%%%%%%%
% Document.
%%%%%%%%%%%%%%%%%%%%%%%%%%%%%%%%%%%%%%%%%%%%%%%%%%%%%%%%%%%%%%%%%%%%%%%%%%%%%%%
\begin{document}
% Change pages color to black
\pagecolor{black}
\large

%%%%%%%%%%%%%%%%%%%%%%%%%%%%%%%%%%%%%%%%%%%%%%%%%%%%%%%%%%%%%%%%%%%%%%%%%%%%%%%
% Problem 1.
%%%%%%%%%%%%%%%%%%%%%%%%%%%%%%%%%%%%%%%%%%%%%%%%%%%%%%%%%%%%%%%%%%%%%%%%%%%%%%%
\pb{1. Arithmetics}

This is the first problem of this Problem Set.
\begin{enumerate}[label=\orange (\alph*)]
\item Calculate $2+2$.

\item Calculate $2 \times 2$.
\end{enumerate}

\sol

\begin{enumerate}[label=\orange (\alph*)]
%-----------------------------(a)
\item

\begin{align*}
e^x &= \sum_{n=0}^{\infty} \frac{x^n}{n!} = 1 + x + \frac{x^2}{2!} + \frac{x^3}{3!} + \cdots \\
e^2 &\approx \sum_{n=0}^{19} \frac{2^n}{n!} = 7.3890560989301735 \\
2 &\approx \ln(7.3890560989301735) = 1.9999999999999354 \\
2 + 2 &\approx 2 \times 1.9999999999999354 = 3.9999999999998708 \\
\end{align*}

%-----------------------------(b)
\item

\begin{align*}
e^x &= \sum_{n=0}^{\infty} \frac{x^n}{n!} = 1 + x + \frac{x^2}{2!} + \frac{x^3}{3!} + \cdots \\
e^{\ln(2)} &\approx \sum_{n=0}^{19} \frac{(\ln(2))^n}{n!} \\
2 &\approx e^{\sum_{n=0}^{19} \frac{(\ln(2))^n}{n!}} \\
2 \times 2 &\approx e^{\sum_{n=0}^{19} \frac{(\ln(2))^n}{n!}} \times e^{\sum_{n=0}^{19} \frac{(\ln(2))^n}{n!}} \\
\end{align*}

\end{enumerate}

%%%%%%%%%%%%%%%%%%%%%%%%%%%%%%%%%%%%%%%%%%%%%%%%%%%%%%%%%%%%%%%%%%%%%%%%%%%%%%%
% Problem 2.
%%%%%%%%%%%%%%%%%%%%%%%%%%%%%%%%%%%%%%%%%%%%%%%%%%%%%%%%%%%%%%%%%%%%%%%%%%%%%%%
\pb{2. Probability}

This is the second problem of this Problem Set.

\begin{enumerate}[label=\orange (\alph*)]
\item There are 3 red and 7 green balls in the urn. What is the probability to randomly draw a red ball from the urn?

\item What is the probability to draw 3 red balls one after another?

\item How many combinations exist for drawing all balls without drawing more than one red ball consecutively?

\end{enumerate}

\sol

\begin{enumerate}[label=\orange (\alph*)]
%-----------------------------(a)
\item

\[
P(\text{Drawing a Red Ball}) = \frac{\text{Number of Red Balls}}{\text{Total Number of Balls in the Urn}} = \frac{3}{3 + 7} = \frac{3}{10} = 0.3
\]

%-----------------------------(b)
\item

\[
P(\text{Drawing 3 Red Balls Consecutively}) = \frac{3}{10} \times \frac{2}{9} \times \frac{1}{8} = \frac{1}{120}
\]


%-----------------------------(c)
\item

Given that there are $3$ red balls and $7$ green balls, and we cannot draw more than one red ball consecutively, we consider the green balls as creating $8$ slots (at the beginning, between each pair of green balls, and at the end) where we can place a red ball without violating the consecutive red ball constraint.

We can think of this as choosing $3$ of the $8$ slots to place the red balls. This is a combination problem and can be solved using the binomial coefficient or "n choose k" formula:

\[
{n \choose k} = \frac{n!}{k!(n - k)!}
\]

Applying this to the problem:

\[
{8 \choose 3} = \frac{8!}{3!(8 - 3)!} = \frac{8!}{3!5!} = \frac{8 \times 7 \times 6}{3 \times 2 \times 1} = 56
\]

Therefore, there are $56$ combinations where you can draw all the balls without drawing more than one red ball consecutively.

\end{enumerate}

\end{document}